%% start of file `template.tex'.
%% Copyright 2006-2015 Xavier Danaux (xdanaux@gmail.com).
%
% This work may be distributed and/or modified under the
% conditions of the LaTeX Project Public License version 1.3c,
% available at http://www.latex-project.org/lppl/.
\documentclass[10pt,a4paper,sans]{moderncv}        % possible options include font size ('10pt', '11pt' and '12pt'), paper size ('a4paper', 'letterpaper', 'a5paper', 'legalpaper', 'executivepaper' and 'landscape') and font family ('sans' and 'roman')
\usepackage[8pt]{extsizes}
\moderncvstyle{classic}
\firstname{Font}
\familyname{Size}

\usepackage{lastpage}
%\rfoot{\addressfont\itshape\textcolor{gray}{\thepage\ / \pageref{LastPage} }}
\rfoot{\addressfont\itshape\textcolor{gray}{\thepage\ / \pageref{LastPage}}}
% moderncv themes
\moderncvstyle{classic}                             % style options are 'casual' (default), 'classic', 'banking', 'oldstyle' and 'fancy'
\moderncvcolor{blue}                               % color options 'black', 'blue' (default), 'burgundy', 'green', 'grey', 'orange', 'purple' and 'red'
%\renewcommand{\familydefault}{\sfdefault}         % to set the default font; use '\sfdefault' for the default sans serif font, '\rmdefault' for the default roman one, or any tex font name
%\nopagenumbers{}                                  % uncomment to suppress automatic page numbering for CVs longer than one page
% character encoding
\usepackage{graphicx}
\usepackage{color}
\usepackage[utf8]{inputenc}                       % if you are not using xelatex ou lualatex, replace by the encoding you are using
%\usepackage{CJKutf8}                              % if you need to use CJK to typeset your resume in Chinese, Japanese or Korean
% adjust the page margins
\usepackage[scale=0.8]{geometry}
%\setlength{\hintscolumnwidth}{3cm}                % if you want to change the width of the column with the dates
%\setlength{\makecvtitlenamewidth}{10cm}           % for the 'classic' style, if you want to force the width allocated to your name and avoid line breaks. be careful though, the length is normally calculated to avoid any overlap with your personal info; use this at your own typographical risks...
% personal data
\name{Adrián}{García García}
\title{Curriculum vitae}                               % optional, remove / comment the line if not wanted
\address{C/Diligencia 4, D 4ºB}{28018 Madrid}{España}% optional, remove / comment the line if not wanted; the "postcode city" and "country" arguments can be omitted or provided empty
\phone[mobile]{639570909}                   % optional, remove / comment the line if not wanted; the optional "type" of the phone can be "mobile" (default), "fixed" or "fax"
%\phone[fixed]{+2~(345)~678~901}
\email{adriagar@ucm.es}                               % optional, remove / comment the line if not wanted
%\homepage{www.johndoe.com}                         % optional, remove / comment the line if not wanted
\social[linkedin]{Adrián García García}                        % optional, remove / comment the line if not wanted
%\social[twitter]{jdoe}                             % optional, remove / comment the line if not wanted
\social[github]{mizadri}                              % optional, remove / comment the line if not wanted
%\extrainfo{additional information}                 % optional, remove / comment the line if not wanted
\photo[64pt][0.4pt]{carnet}                       % optional, remove / comment the line if not wanted; '64pt' is the height the picture must be resized to, 0.4pt is the thickness of the frame around it (put it to 0pt for no frame) and 'picture' is the name of the picture file
%\quote{Some quote}                                 % optional, remove / comment the line if not wanted

% bibliography adjustements (only useful if you make citations in your resume, or print a list of publications using BibTeX)
%   to show numerical labels in the bibliography (default is to show no labels)
\makeatletter\renewcommand*{\bibliographyitemlabel}{\@biblabel{\arabic{enumiv}}}\makeatother
%   to redefine the bibliography heading string ("Publications")
%\renewcommand{\refname}{Articles}

% bibliography with mutiple entries
%\usepackage{multibib}
%\newcites{book,misc}{{Books},{Others}}
%----------------------------------------------------------------------------------
%            content
%----------------------------------------------------------------------------------
\begin{document}
%\begin{CJK*}{UTF8}{gbsn}                          % to typeset your resume in Chinese using CJK
%-----       resume       ---------------------------------------------------------
\makecvtitle

\section{Formación académica}
%\cventry{year--year}{Degree}{Institution}{City}{\textit{Grade}}{Description}  % arguments 3 to 6 can be left empty
\cventry{2011--2016}{Grado en Ingeniería Informática}{UCM}{Madrid}{\textit{Media: 7,82}}{} % arguments 3 to 6 can be left empty
\cventry{2016--2018}{Máster en Ingeniería Informática}{UCM}{Madrid}{\textit{Media: 8,9}}{}
\cventry{2017--actualidad}{Doctorando en Ingeniería Informática}{DACYA}{Madrid}{}{Contrato predoctoral UCM de personal investigador en formación.}

% \section{Master thesis}
% \cvitem{title}{\emph{Title}}
% \cvitem{supervisors}{Supervisors}
% \cvitem{description}{Short thesis abstract}

\section{Experiencia profesional}
% \cventry{year--year}{Job title}{Employer}{City}{}{Description line 1\newline{}Description line 2}
\cventry{Abr. 2016- Ago. 2016}{Becario de desarrollo software}{SATEC}{Madrid}{}{ Desarrollo de aplicaciones web de carácter empresarial con .NET MVC, Razor y jQuery UI, en concreto se participó en la creación de una aplicación web que gestionaba el parque móvil del estado.}
\cventry{Nov. 2016- Marzo 2017.}{Beca de colaboración con departamentos}{DACYA}{Madrid}{}{Desarrollo y estudio de un planificador en el kernel Linux consciente de la contención de recursos compartidos en sistemas asimétricos. Otorgada por el Ministerio de Educación, Cultura y Deporte.}
\cventry{Feb. 2017- Marzo 2017.}{Administrador de sistemas}{DACYA}{Madrid}{}{Administración de sistemas Linux y mantenimiento de la infraestructura física y virtualizada.}

\section{Artículos de investigación}
\begin{itemize}
\item A. Garcia-Garcia, J. C. Saez, and M. Prieto-Matias. 2017. "Delivering Fairness on Asymmetric Multicore Systems via Contention-Aware Scheduling". 5th Workshop on Runtime and Operating Systems for the Many-core Era. Euro-Par 2017: Parallel Processing Workshops. Springer International Publishing 2018. 610—622.
\smallskip
\item A. Garcia-Garcia, J. C. Saez, and M. Prieto-Matias. 2018. "Contention-Aware Fair Scheduling for Asymmetric Single-ISA Multicore Systems". IEEE Transactions on Computers. 67, 12 (Dec 2018), 1703–1719.
\smallskip
\item (Presentador) A. Garcia-Garcia, J. C. Saez, F. Castro, M. Prieto-Matias. 2019. "LFOC: A Lightweight Fairness-Oriented Cache Clustering Policy for Commodity Multicores". In Proceedings of the 48th International Conference on Parallel Processing (ICPP 2019).
\end{itemize}

\section{Premios}
\cventry{Junio de 2016}{Matrícula de honor en Trabajo de Fin de Grado}{Facultad de Informática, UCM}{}{}{\begin{itemize}
\item \textbf{Título}: Soporte de sistema operativo para ahorro de energía en plataformas móviles con procesadores multicore asimétricos.
\item \textbf{Director}: Juan Carlos Sáez Alcaide.
\item \textbf{Departamento de Arquitectura de Computadores y Automática}.
\end{itemize}}
\cventry{2011--2016}{Matriculas de honor del Grado en Ingeniería Informática}{UCM}{Madrid}{}{\begin{itemize}
\item Sistemas operativos
\item Ampliación de sistemas operativos y redes
\item Arquitectura de computadores
\item Arquitectura interna de Linux y Android
\item Redes y seguridad
\item Programación paralela para móviles y multicores
\item Bases de datos
\item Evaluación de configuraciones
\end{itemize}}

\section{Otros méritos}
\cventry{Sept. 2016}{Inglés}{Nivel C1}{}{}{\begin{itemize}
\item Certificados de la Universidad de Cambridge.
\begin{itemize}
  \item {\color{blue}\href{https://drive.google.com/open?id=0B0Q9HPrQCrfganJjYmFWdlFoanM}{First Certificate in English}} (FCE), Madrid, Abril de 2012.
  \item {\color{blue}\href{https://drive.google.com/open?id=0B0Q9HPrQCrfgcGFJX3VPTGFyUE0}{Certificate in Advanced English}} (CAE), Madrid, Septiembre de 2016.
\end{itemize}
\item Formación en la academia Dickens (2011, 2015, 2017-actualidad).
\item Estancia de 1 mes en Athenry (Irlanda) con una familia nativa durante la que se realizaba un
curso intensivo de inglés (Agosto de 2010): {\color{blue}\href{https://drive.google.com/open?id=0B0Q9HPrQCrfgSmRCM1NQV2lfdGc}{Certificado}}.
\end{itemize}}
\cventry{Noviembre 2016}{Big Data: análisis de datos con python}{UCM}{Madrid}{}{Curso de formación continua de la UCM que enseñaba a obtener datos de páginas web, Twitter y ficheros en distintos formatos (.csv, .xls, .txt, etc.) para luego trabajar con ellos usando diferentes librerías científicas de python: numpy, matplotlib y pandas. {\color{blue}\href{https://drive.google.com/open?id=0B0Q9HPrQCrfgbFMwUUdOQXd1QWs}{Certificado}}.}
\section{Otros aspectos}
\cventry{Conocimientos}{Lenguajes de programación}{}{}{}{\begin{itemize}
\item \textbf{Python}: pandas, numpy, matplotlib, Telegram API.
\smallskip
\item \textbf{C}: Módulos del kernel,  Sockets, Tuberías.
\smallskip
\item \textbf{C++}: OpenMP, MARE, OpenGL.
\smallskip
\item \textbf{C\#}: ASP.NET MVC, Razor, jQuery UI.
\smallskip
\item \textbf{Java}: Aplicaciones Android ({\color{blue}\href{https://github.com/android-fridge/thefridge}{TheFridge}}), Hibernate, JSP, Maven.
\smallskip
\item \textbf{Web}: PHP, JavaScript (jQuery), CSS (Boostrap).
\smallskip\end{itemize}}
\cventry{}{Otras tecnologías}{}{}{}{\begin{itemize}
\item \textbf{Software de control de versiones} ({\color{blue}\href{https://github.com/mizadri/}{GitHub}})
\smallskip
\item \textbf{Bases de datos}: SQL, MongoDB, Neo4J.
\smallskip
\item \textbf{Big Data}: Map Reduce, Spark, Hadoop File System.
\smallskip\item \textbf{Machine Learning} ({\color{blue}\href{https://github.com/mizadri/big-data}{Implementaciones en python}}):
  \begin{itemize}
    \item Árboles de clasificación ID3.
    \item Índice completo (compresión de datos con variable-bytes, elias-gamma o elias-delta).
    \item Índice vectorial.
    \item Clasificador Bayesiano Simple (Naive Bayes).
  \end{itemize}
  \medskip
\item \textbf{Administración de redes y virtualización}
\begin{itemize}
  \item Configuración de redes IPv4 e IPv6 (configuración persistente con \textit{vtysh}).
  \item Administración de máquinas via SSH.
  \item Gestión de máquinas virtuales con kvm (\textit{virsh} y \textit{virt-manager}).
  \item Instalación de entornos Docker.
\end{itemize}
\medskip
\item \textbf{Edición de textos en LaTeX} ({\color{blue}\href{https://drive.google.com/open?id=0B0Q9HPrQCrfgdng1eHFyU1JPY1U}{Memoria de TFG}} o {\color{blue}\href{https://github.com/mizadri/LaTeX}{este CV}})
\end{itemize}}

% \section{Computer skills}
% \cvdoubleitem{category 1}{XXX, YYY, ZZZ}{category 4}{XXX, YYY, ZZZ}
% \cvdoubleitem{category 2}{XXX, YYY, ZZZ}{category 5}{XXX, YYY, ZZZ}
% \cvdoubleitem{category 3}{XXX, YYY, ZZZ}{category 6}{XXX, YYY, ZZZ}

% \section{References}
% \begin{cvcolumns}
%   \cvcolumn{Category 1}{\begin{itemize}\item Person 1\item Person 2\item Person 3\end{itemize}}
%   \cvcolumn{Category 2}{Amongst others:\begin{itemize}\item Person 1, and\item Person 2\end{itemize}(more upon request)}
%   \cvcolumn[0.5]{All the rest \& some more}{\textit{That} person, and \textbf{those} also (all available upon request).}
% \end{cvcolumns}

% Publications from a BibTeX file without multibib
%  for numerical labels: \renewcommand{\bibliographyitemlabel}{\@biblabel{\arabic{enumiv}}}% CONSIDER MERGING WITH PREAMBLE PART
%  to redefine the heading string ("Publications"): \renewcommand{\refname}{Articles}
\nocite{*}
\bibliographystyle{plain}
\bibliography{publications}                        % 'publications' is the name of a BibTeX file

% Publications from a BibTeX file using the multibib package
%\section{Publications}
%\nocitebook{book1,book2}
%\bibliographystylebook{plain}
%\bibliographybook{publications}                   % 'publications' is the name of a BibTeX file
%\nocitemisc{misc1,misc2,misc3}
%\bibliographystylemisc{plain}
%\bibliographymisc{publications}                   % 'publications' is the name of a BibTeX file


%\clearpage\end{CJK*}                              % if you are typesetting your resume in Chinese using CJK; the \clearpage is required for fancyhdr to work correctly with CJK, though it kills the page numbering by making \lastpage undefined
\end{document}