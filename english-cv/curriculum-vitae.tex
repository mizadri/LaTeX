%% start of file `template.tex'.
%% Copyright 2006-2015 Xavier Danaux (xdanaux@gmail.com).
%
% This work may be distributed and/or modified under the
% conditions of the LaTeX Project Public License version 1.3c,
% available at http://www.latex-project.org/lppl/.
\documentclass[10pt,a4paper,sans]{moderncv}        % possible options include font size ('10pt', '11pt' and '12pt'), paper size ('a4paper', 'letterpaper', 'a5paper', 'legalpaper', 'executivepaper' and 'landscape') and font family ('sans' and 'roman')
\usepackage[8pt]{extsizes}
\moderncvstyle{classic}
\firstname{Font}
\familyname{Size}

\usepackage{lastpage}
%\rfoot{\addressfont\itshape\textcolor{gray}{\thepage\ / \pageref{LastPage} }}
\rfoot{\addressfont\itshape\textcolor{gray}{\thepage\ / \pageref{LastPage}}}
% moderncv themes
\moderncvstyle{classic}                             % style options are 'casual' (default), 'classic', 'banking', 'oldstyle' and 'fancy'
\moderncvcolor{blue}                               % color options 'black', 'blue' (default), 'burgundy', 'green', 'grey', 'orange', 'purple' and 'red'
%\renewcommand{\familydefault}{\sfdefault}         % to set the default font; use '\sfdefault' for the default sans serif font, '\rmdefault' for the default roman one, or any tex font name
%\nopagenumbers{}                                  % uncomment to suppress automatic page numbering for CVs longer than one page
% character encoding
\usepackage{graphicx}
\usepackage{color}
\usepackage[utf8]{inputenc}                       % if you are not using xelatex ou lualatex, replace by the encoding you are using
%\usepackage{CJKutf8}                              % if you need to use CJK to typeset your resume in Chinese, Japanese or Korean
% adjust the page margins
\usepackage[scale=0.8]{geometry}
%\setlength{\hintscolumnwidth}{3cm}                % if you want to change the width of the column with the dates
%\setlength{\makecvtitlenamewidth}{10cm}           % for the 'classic' style, if you want to force the width allocated to your name and avoid line breaks. be careful though, the length is normally calculated to avoid any overlap with your personal info; use this at your own typographical risks...
% personal data
\name{Adrian}{Garcia Garcia}
\title{Curriculum vitae}                               % optional, remove / comment the line if not wanted
\phone[mobile]{639570909}                   % optional, remove / comment the line if not wanted; the optional "type" of the phone can be "mobile" (default), "fixed" or "fax"
%\phone[fixed]{+2~(345)~678~901}
\email{mizadri.gg@gmail.com}                               % optional, remove / comment the line if not wanted
%\homepage{www.johndoe.com}                         % optional, remove / comment the line if not wanted
\social[linkedin]{mizadri}                        % optional, remove / comment the line if not wanted
%\social[twitter]{jdoe}                             % optional, remove / comment the line if not wanted
\social[github]{mizadri}                              % optional, remove / comment the line if not wanted
%\extrainfo{additional information}                 % optional, remove / comment the line if not wanted
%\photo[64pt][0.4pt]{carnet}                       % optional, remove / comment the line if not wanted; '64pt' is the height the picture must be resized to, 0.4pt is the thickness of the frame around it (put it to 0pt for no frame) and 'picture' is the name of the picture file
%\quote{Some quote}                                 % optional, remove / comment the line if not wanted

% bibliography adjustements (only useful if you make citations in your resume, or print a list of publications using BibTeX)
%   to show numerical labels in the bibliography (default is to show no labels)
\makeatletter\renewcommand*{\bibliographyitemlabel}{\@biblabel{\arabic{enumiv}}}\makeatother
%   to redefine the bibliography heading string ("Publications")
%\renewcommand{\refname}{Articles}

% bibliography with mutiple entries
%\usepackage{multibib}
%\newcites{book,misc}{{Books},{Others}}
%----------------------------------------------------------------------------------
%            content
%----------------------------------------------------------------------------------
\begin{document}
%\begin{CJK*}{UTF8}{gbsn}                          % to typeset your resume in Chinese using CJK
%-----       resume       ---------------------------------------------------------
\makecvtitle

\section{Description} I am a highly curious and passion driven individual that is always looking to grow as a professional and as a person. My dream job offers interesting technical challenges that are faced through teamwork and allow me to learn new technologies, where my researching and learning skills could be useful. No matter the place I can find a way there, specially if the software we create actually makes a difference.

\section{Education}
%\cventry{year--year}{Degree}{Institution}{City}{\textit{Grade}}{Description}  % arguments 3 to 6 can be left empty
\cventry{2017--2021}{PhD in Computer Science Engineering}{Complutense University of Madrid}{}{\textit{cum laude mention} {\color{blue}\href{https://drive.google.com/file/d/1WwCfr2Mp8xMYrpEOq5sLz3ep9YLhmNLN/view?usp=sharing}{Thesis manuscript}}}{}
\cventry{2016--2018}{Master's Degree in Computer Science Engineering}{Complutense University of Madrid}{}{\textit{Current average: 9,12.} {\color{blue}\href{https://drive.google.com/open?id=0B0Q9HPrQCrfgNFNnQnNJeVNxT00}{Academic Report}}}{}
\cventry{2011--2016}{Bachelor's Degree in Computer Science Engineering}{Complutense University of Madrid}{}{\textit{Average: 7,82 over 10.} {\color{blue}\href{https://drive.google.com/open?id=0B0Q9HPrQCrfgUzllSEpXd1BSLWc}{Academic Report}}}{} % arguments 3 to 6 can be left empty

\section{Employment history}
% \cventry{year--year}{Job title}{Employer}{City}{}{Description line 1\newline{}Description line 2}
\cventry{Apr'18-Dec'21}{PhD scolarship (UCM)}{Department of Compuer Architecture and Automatics}{Madrid}{}{"Contention-aware scheduling and resource management for emerging multicore architectures": the main proposals were implemented in C Linux kernel modules, but most of the experimental analysis and performance metrics was done in python (pandas/matlotlib), as well as some cluster parallelization (ipyparallel)}.
\cventry{Oct'17-Oct'18}{System Administrator}{UCM: Department of Compuer Architecture and Automatics}{Madrid}{}{Administrated multiple Linux systems and maintained the physical and virtualized infraestructure (kvm).}
\cventry{Nov'16-Jun'17}{Collaboration fellowship by Spanish MECD}{Seven months researching and working for the Department of Computer Architecture and Automatics}{}{Madrid}{}{}
\cventry{Apr'16-Aug'16}{Internship in Software Developement}{SATEC}{Madrid}{}{Developed web applications in C\# (.NET MVC), Razor and jQuery UI. Specifically, I contributed to build a web application for the government car management system.}

\section{Research articles}
\begin{itemize}
\item Garcia-Garcia, A., Saez, J.C., Prieto, M.: "Delivering Fairness on Asymmetric Multicore Systems via Contention-Aware Scheduling". 5th Workshop on Runtime and Operating Systems for the Many-core Era. Euro-Par 2017: Parallel Processing Workshops. Springer International Publishing 2018. 610—622. 
\smallskip
\item Garcia-Garcia, A., Saez, J.C., Prieto, M.: “Contention-Aware Fair Scheduling for Asymmetric Single-ISA Multicore Systems”. IEEE Transactions on Computers, 67 (12): 1703-1719. (2018). IF (JCR): 3.131. ISSN: 0018-9340; DOI: 10.1109/TC.2018.2836418. Impact index of \textbf{JCR Q1} by Computer Science, Hardware and Architecture (2018).
\smallskip
\item Presented this article in the ICPP 2019 that took place in Kyoto: Garcia-Garcia, A., Saez, J.C., Castro, F. Prieto-Matias, M.: “LFOC: A Lightweight Fairness-Oriented Cache Clustering Policy for Commodity Multicores”. In Proceedings of the 48th International Conference on Parallel Processing (ICPP '19): 14:1-14:10. (2019).  ISBN: 978-1-4503-6295-5; DOI: 10.1145/3337821.3337925. \textbf{Ranked Class II} in GII-GRIN-SCIE (2018)
\smallskip
\item Garcia-Garcia, A., Saez, J.C., Risco-Martin, J.L., Prieto, M.. “PBBCache: An open-source parallel simulator for rapid prototyping and evaluation of cache-partitioning and cache-clustering policies”. Journal of Computational Science, 42:101102, 2020. https://doi.org/10.1016/j.jocs.2020.101102. Impact index of \textbf{JCR Q1} by Computer Science, Theory and Methods (2020). Available on https://github.com/pbbcache/cachesim
\end{itemize}


\section{Achievements}
\cventry{December, 2021}{Obtained my PhD with cum laude mention}{}{}{}{}
\cventry{June, 2016}{Graduated with special mention in the Final Project of my Bachelor's Degree}{}{}{}{}
\cventry{2011--2016}{Subjects passed with honors on the Bachelor's Degree in Computer Science Engineering}{}{}{}{\begin{itemize}
\item Operating Systems
\item Advanced Operating Systems and Networks
\item Computer Architecture
\item Linux and Android Internals
\item Computer Networks Security
\item Mobile and multicore parallel programming
\item Databases
\item Evaluation of Computer Systems
\end{itemize}}


\section{Additional information}
\cventry{12-30-1993}{Spanish}{Native}{}{}{}
\cventry{Sept. 2016}{English}{Level: C1}{}{}{\begin{itemize}
\item Cambridge certificates:
\begin{itemize}
  \item {\color{blue}\href{https://drive.google.com/open?id=0B0Q9HPrQCrfganJjYmFWdlFoanM}{First Certificate in English}} (FCE), Madrid, April 2012.
  \item {\color{blue}\href{https://drive.google.com/open?id=0B0Q9HPrQCrfgcGFJX3VPTGFyUE0}{Certificate in Advanced English}} (CAE), Madrid, September 2016.
\end{itemize}
\end{itemize}}
\cventry{Nov. 2016}{Big Data: analysis with python}{Complutense University of Madrid}{}{}{Continious formation course of the UCM that teached to gather data from web pages, Twitter and different file formats (csv, xls, txt, etc.) in order to parse them and work with them using python libraries: numpy, matplotlib and pandas. {\color{blue}\href{https://drive.google.com/open?id=0B0Q9HPrQCrfgbFMwUUdOQXd1QWs}{Certificate}}.}


\section{Other skills}
\cventry{}{Programming languages}{}{}{}{\begin{itemize}
\item \textbf{Python}: scikit-learn, pandas, numpy, matplotlib, ipyparallel.
\smallskip
\item \textbf{C/C++}: Kernel modules (drivers/scheduling), OpenGL, OpenMP/MPI, Sockets.
\smallskip
\item \textbf{Java}: Android app developement ({\color{blue}\href{https://github.com/android-fridge/thefridge}{TheFridge}}), Hibernate, JSP, Maven.
\smallskip
\item \textbf{Web}: PHP, C\# ASP.NET MVC (Razor), JavaScript, CSS (Boostrap).
\smallskip
\item \textbf{Functional Programming}: Erlang, Maude.
\smallskip\end{itemize}}
\cventry{}{Other technologies}{}{}{}{\begin{itemize}
\item \textbf{Databases}: SQL, MongoDB, Neo4J.
\smallskip
\item \textbf{Big Data}: Spark, Map Reduce, Hadoop File System.
\smallskip\item \textbf{Machine Learning} ({\color{blue}\href{https://github.com/mizadri/big-data}{python implementations}}):
  \begin{itemize}
    \item Classification trees (ID3).
    \item Complete Index (data compression: variable-bytes, elias-gamma or elias-delta).
    \item Vectorial Index.
    \item Naive Bayes.
  \end{itemize}
\item \textbf{Network management and virtualization}
\begin{itemize}
  \item Persistent network configuration IPv4/IPv6 (\textit{vtysh}).
  \item Virtual machine management in servers with kvm (\textit{virsh} y \textit{virt-manager}).
  \item Basic usage of docker to create and deploy images.
  \item Firewall deployment and network analysis skills (wireshark/IDS)
\end{itemize}
\smallskip
\end{itemize}}


\section{References}
% \cventry{year--year}{Job title}{Employer}{City}{}{Description line 1\newline{}Description line 2}
\cventry{}{Manuel Prieto Matías}{Associate Professor at Complutense Univerity of Madrid}{mpmatias@ucm.es}{Research group website {\color{blue}\href{https://artecs.dacya.ucm.es/}{(ArTeCs)}}}{}
\cventry{}{Juan Carlos Sáez Alcaide}{Associate Professor at Complutense Univerity of Madrid}{jcsaezal@ucm.es}{Research group website {\color{blue}\href{https://artecs.dacya.ucm.es/}{(ArTeCs)}}}{}


\nocite{*}
\bibliographystyle{plain}
\bibliography{publications}

\end{document}