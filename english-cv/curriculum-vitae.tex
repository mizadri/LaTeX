%% start of file `template.tex'.
%% Copyright 2006-2015 Xavier Danaux (xdanaux@gmail.com).
%
% This work may be distributed and/or modified under the
% conditions of the LaTeX Project Public License version 1.3c,
% available at http://www.latex-project.org/lppl/.
\documentclass[10pt,a4paper,sans]{moderncv}        % possible options include font size ('10pt', '11pt' and '12pt'), paper size ('a4paper', 'letterpaper', 'a5paper', 'legalpaper', 'executivepaper' and 'landscape') and font family ('sans' and 'roman')
\usepackage[8pt]{extsizes}
\moderncvstyle{classic}
\firstname{Font}
\familyname{Size}

\usepackage{lastpage}
%\rfoot{\addressfont\itshape\textcolor{gray}{\thepage\ / \pageref{LastPage} }}
\rfoot{\addressfont\itshape\textcolor{gray}{\thepage\ / \pageref{LastPage}}}
% moderncv themes
\moderncvstyle{classic}                             % style options are 'casual' (default), 'classic', 'banking', 'oldstyle' and 'fancy'
\moderncvcolor{blue}                               % color options 'black', 'blue' (default), 'burgundy', 'green', 'grey', 'orange', 'purple' and 'red'
%\renewcommand{\familydefault}{\sfdefault}         % to set the default font; use '\sfdefault' for the default sans serif font, '\rmdefault' for the default roman one, or any tex font name
%\nopagenumbers{}                                  % uncomment to suppress automatic page numbering for CVs longer than one page
% character encoding
\usepackage{graphicx}
\usepackage{color}
\usepackage[utf8]{inputenc}                       % if you are not using xelatex ou lualatex, replace by the encoding you are using
%\usepackage{CJKutf8}                              % if you need to use CJK to typeset your resume in Chinese, Japanese or Korean
% adjust the page margins
\usepackage[scale=0.8]{geometry}
%\setlength{\hintscolumnwidth}{3cm}                % if you want to change the width of the column with the dates
%\setlength{\makecvtitlenamewidth}{10cm}           % for the 'classic' style, if you want to force the width allocated to your name and avoid line breaks. be careful though, the length is normally calculated to avoid any overlap with your personal info; use this at your own typographical risks...
% personal data
\name{Adrian}{Garcia Garcia}
\title{Curriculum vitae}                               % optional, remove / comment the line if not wanted
\address{C/Diligencia 4, D 4ºB}{28018 Madrid}{Spain}% optional, remove / comment the line if not wanted; the "postcode city" and "country" arguments can be omitted or provided empty
\phone[mobile]{639570909}                   % optional, remove / comment the line if not wanted; the optional "type" of the phone can be "mobile" (default), "fixed" or "fax"
%\phone[fixed]{+2~(345)~678~901}
\email{adriagar@ucm.es}                               % optional, remove / comment the line if not wanted
%\homepage{www.johndoe.com}                         % optional, remove / comment the line if not wanted
\social[linkedin]{Adrián García García}                        % optional, remove / comment the line if not wanted
%\social[twitter]{jdoe}                             % optional, remove / comment the line if not wanted
\social[github]{mizadri}                              % optional, remove / comment the line if not wanted
%\extrainfo{additional information}                 % optional, remove / comment the line if not wanted
\photo[64pt][0.4pt]{carnet}                       % optional, remove / comment the line if not wanted; '64pt' is the height the picture must be resized to, 0.4pt is the thickness of the frame around it (put it to 0pt for no frame) and 'picture' is the name of the picture file
%\quote{Some quote}                                 % optional, remove / comment the line if not wanted

% bibliography adjustements (only useful if you make citations in your resume, or print a list of publications using BibTeX)
%   to show numerical labels in the bibliography (default is to show no labels)
\makeatletter\renewcommand*{\bibliographyitemlabel}{\@biblabel{\arabic{enumiv}}}\makeatother
%   to redefine the bibliography heading string ("Publications")
%\renewcommand{\refname}{Articles}

% bibliography with mutiple entries
%\usepackage{multibib}
%\newcites{book,misc}{{Books},{Others}}
%----------------------------------------------------------------------------------
%            content
%----------------------------------------------------------------------------------
\begin{document}
%\begin{CJK*}{UTF8}{gbsn}                          % to typeset your resume in Chinese using CJK
%-----       resume       ---------------------------------------------------------
\makecvtitle

\section{Academic qualifications}
%\cventry{year--year}{Degree}{Institution}{City}{\textit{Grade}}{Description}  % arguments 3 to 6 can be left empty
\cventry{2011--2016}{Bachelor's Degree in Computer Science Engineering}{Complutense University of Madrid}{}{\textit{Average: 7,82}}{} % arguments 3 to 6 can be left empty
\cventry{2016--Now}{Master's Degree in Computer Science Engineering}{Complutense University of Madrid}{}{\textit{Current average: 9,12}}{}


% \section{Master thesis}
% \cvitem{title}{\emph{Title}}
% \cvitem{supervisors}{Supervisors}
% \cvitem{description}{Short thesis abstract}
\section{Achievements}
\cventry{June, 2016}{Graduated with special mention in the Final Project}{Complutense University of Madrid}{}{}{\begin{itemize}
\item \textbf{Title}: Operating system support for improving energy efficiency in asymmetric multicore systems.
\item \textbf{Director}: Juan Carlos Sáez Alcaide.
\item \textbf{Department of Computers' architecture and automatic}.
\end{itemize}}
\cventry{2011--2016}{Subjects passed with honors on the Bachelor's Degree in Computer Science Engineering}{Complutense University of Madrid}{}{}{\begin{itemize}
\item Operating Systems
\item Advanced Operating Systems and Networks
\item Computer Architecture
\item Linux and Android kernel architecture
\item Networks and Security
\item Mobile and multicore parallel programming
\item Databases
\item Cofiguration assessment
\end{itemize}}

\section{Work experience}
% \cventry{year--year}{Job title}{Employer}{City}{}{Description line 1\newline{}Description line 2}
\cventry{Apr.-Aug. (2016)}{Internship in Software Developement}{SATEC}{Madrid}{}{Developement of web applications in C\# (.NET MVC), Razor and jQuery UI. Specifically, I contributed to build a web application that managed the government car management system.}
\cventry{Nov. 2016}{Seven months in a colaboration scholarship with university departments}{Department of Compuer Architecture and Automatics}{Madrid}{}{Study and developement of a Linux kernel scheduler aware of the effects of shared-resource contention in asymmetric multicore systems. Awarded by the Spanish Ministry of Education, Culture and Sport.}
\cventry{Feb. 2017-Now}{System Administrator}{Department of Compuer Architecture and Automatics}{Madrid}{}{Administration of Linux systems and maintenance of the physical and virtualized infraestructure.}
\cventry{Aug. 2016}{Deployment and developement in Wordpress of a website for a Spanish political party {\color{blue}\href{www.ahoraarganda.es}{Ahora Arganda}}}{}{}{}{}

\section{Additional information}
\cventry{12-30-1993}{Spanish}{Native}{}{}{}
\cventry{Sept. 2016}{English}{Level: C1}{}{}{\begin{itemize}
\item Cambridge English Certificates
\begin{itemize}
  \item {\color{blue}\href{https://drive.google.com/open?id=0B0Q9HPrQCrfganJjYmFWdlFoanM}{First Certificate in English}} (FCE), Madrid, April 2012.
  \item {\color{blue}\href{https://drive.google.com/open?id=0B0Q9HPrQCrfgcGFJX3VPTGFyUE0}{Certificate in Advanced English}} (CAE), Madrid, September 2016.
\end{itemize}
\item English courses with native teachers in Dicken's academy (2011 y 2015).
\item I stayed one month with a foster family in Athenry (Ireland) while I studied an intensive english course on the mornings (August 2010): {\color{blue}\href{https://drive.google.com/open?id=0B0Q9HPrQCrfgSmRCM1NQV2lfdGc}{Certificate}}.
\end{itemize}}
\cventry{Nov. 2016}{Course in Big Data: data analysis with python}{Complutense University of Madrid}{}{}{Continious formation course of the UCM that teached to gather data from web pages, Twitter and different file formats (csv, xls, txt, etc.) in order to parse them and work with them using different python libraries: numpy, matplotlib and pandas.{\color{blue}\href{https://drive.google.com/open?id=0B0Q9HPrQCrfgbFMwUUdOQXd1QWs}{Certificate}}.}
\section{Other skills}
\cventry{Known technologies}{}{}{}{}{\begin{itemize}
\item \textbf{Python}
\begin{itemize}
  \item Pandas
  \item Numpy
  \item Matplotlib
  \item Telegram API
\end{itemize}
\item \textbf{C}
\begin{itemize}
  \item Linux kernel modules
  \item Sockets
  \item Pipes
\end{itemize}
\item \textbf{C++}
\begin{itemize}
  \item OpenMP
  \item MARE
  \item OpenGL
\end{itemize}
\item \textbf{C\#}
\begin{itemize}
  \item ASP.NET MVC
  \item Razor
  \item jQuery UI
\end{itemize}
\item \textbf{Android apps} ({\color{blue}\href{https://github.com/android-fridge/thefridge}{TheFridge}})
\item \textbf{Java}
\begin{itemize}
  \item Hibernate
  \item JSP
  \item Maven
\end{itemize}
\item \textbf{PHP}
\item \textbf{JavaScript} (jQuery)
\item \textbf{CSS} (Boostrap)
\item \textbf{Bash}
\medskip
\end{itemize}}
\cventry{}{Other technologies}{}{}{}{\begin{itemize}
\item \textbf{Databases}
  \begin{itemize}
  \item SQL
  \item MongoDB
  \item Neo4J
  \item XML
  \item Hibernate
\end{itemize}
\medskip
\item \textbf{Big Data}
\begin{itemize}
  \item Map Reduce
  \item Spark
\end{itemize}
\medskip
\item \textbf{Machine Learning} ({\color{blue}\href{https://github.com/mizadri/big-data}{python implementations}}):
  \begin{itemize}
    \item ID3 classification trees.
    \item Complete Index (Data compressions: variable-bytes, elias-gamma or elias-delta).
    \item Vectorial Index.
    \item Naive Bayes.
  \end{itemize}
  \medskip
\end{itemize}}
\cventry{}{}{}{}{}{\begin{itemize}
\item \textbf{Network administration and virtualization}
\begin{itemize}
  \item IPv4 e IPv6 network configuration. Persistent with \textit{vtysh}(similar to CISCO syntax but Open Source).
  \item System administration via SSH.
  \item Virtual Machine management with kvm (\textit{virsh} and \textit{virt-manager}).
  \item Docker configuration and deployment.
\end{itemize}
\medskip
\item \textbf{LaTeX for Scientific papers and reports} ({\color{blue}\href{https://drive.google.com/open?id=0B0Q9HPrQCrfgdng1eHFyU1JPY1U}{Final Project}} or {\color{blue}\href{https://github.com/mizadri/LaTeX}{this CV}})
\item \textbf{Version control software} ({\color{blue}\href{https://github.com/mizadri/}{GitHub}})
\item Driving license.
\medskip
\end{itemize}}

\nocite{*}
\bibliographystyle{plain}
\bibliography{publications}

\end{document}